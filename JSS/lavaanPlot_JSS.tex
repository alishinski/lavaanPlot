\documentclass[
]{jss}

\usepackage[utf8]{inputenc}

\providecommand{\tightlist}{%
  \setlength{\itemsep}{0pt}\setlength{\parskip}{0pt}}

\author{
Alex Lishinski\\University of Tennessee
}
\title{\pkg{lavaanPlot}: An R package for plotting structural equation models}

\Plainauthor{Alex Lishinski}
\Plaintitle{lavaanPlot: An R package for plotting structural equation models}
\Shorttitle{\pkg{lavaanPlot}: SEM plotting}

\Abstract{
The lavaan package is an excellent package for structural equation
models, and the DiagrammeR package is an excellent package for producing
nice looking graph diagrams. As of right now, the lavaan package has no
built in plotting functions for models, and the available options from
external packages don't look as nice and aren't as easy to use as
DiagrammeR, in my opinion. Of course, you can use DiagrammeR to build
path diagrams for your models, but it requires you to build the diagram
specification manually. This package exists to streamline that process,
allowing you to plot your lavaan models directly, without having to
translate them into the DOT language specification that DiagrammeR uses.
}

\Keywords{keywords, not capitalized, \proglang{Java}}
\Plainkeywords{keywords, not capitalized, Java}

%% publication information
%% \Volume{50}
%% \Issue{9}
%% \Month{June}
%% \Year{2012}
%% \Submitdate{}
%% \Acceptdate{2012-06-04}

\Address{
    Alex Lishinski\\
    University of Tennessee\\
    First line\\
Second line\\
  E-mail: \email{name@company.com}\\
  URL: \url{http://rstudio.com}\\~\\
  }

% Pandoc citation processing

% Pandoc header

\usepackage{amsmath}

\begin{document}

\hypertarget{introduction}{%
\section{Introduction}\label{introduction}}

This template demonstrates some of the basic LaTeX that you need to know
to create a JSS article.

\hypertarget{code-formatting}{%
\subsection{Code formatting}\label{code-formatting}}

In general, don't use Markdown, but use the more precise LaTeX commands
instead:

\begin{itemize}
\item
  \proglang{Java}
\item
  \pkg{plyr}
\end{itemize}

One exception is inline code, which can be written inside a pair of
backticks (i.e., using the Markdown syntax).

If you want to use LaTeX commands in headers, you need to provide a
\texttt{short-title} attribute. You can also provide a custom identifier
if necessary. See the header of Section \ref{r-code} for example.

\section[R code]{\proglang{R} code}\label{r-code}

Can be inserted in regular R markdown blocks.

\begin{CodeChunk}
\begin{CodeInput}
R> x <- 1:10
R> x
\end{CodeInput}
\begin{CodeOutput}
 [1]  1  2  3  4  5  6  7  8  9 10
\end{CodeOutput}
\end{CodeChunk}

\subsection[Features specific to rticles]{Features specific to
\pkg{rticles}}\label{features-specific-to}

\begin{itemize}
\tightlist
\item
  Adding short titles to section headers is a feature specific to
  \pkg{rticles} (implemented via a Pandoc Lua filter). This feature is
  currently not supported by Pandoc and we will update this template if
  \href{https://github.com/jgm/pandoc/issues/4409}{it is officially
  supported in the future}.
\item
  Using the \texttt{\textbackslash{}AND} syntax in the \texttt{author}
  field to add authors on a new line. This is a specific to the
  \texttt{rticles::jss\_article} format.
\end{itemize}



\end{document}

